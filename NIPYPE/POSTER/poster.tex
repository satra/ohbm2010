\documentclass[portrait,final]{baposter}
%\documentclass[a4shrink,portrait,final]{baposter}
% Usa a4shrink for an a4 sized paper.

\tracingstats=2

\usepackage{times}
\usepackage{calc}
\usepackage{graphicx}
\usepackage{amsmath}
\usepackage{amssymb}
\usepackage{relsize}
\usepackage{multirow}
\usepackage{bm}

\usepackage{graphicx}
\usepackage{wrapfig}
\usepackage{multicol}

\usepackage{pgfbaselayers}
\pgfdeclarelayer{background}
\pgfdeclarelayer{foreground}
\pgfsetlayers{background,main,foreground}

\usepackage{helvet}
%\usepackage{bookman}
\usepackage{palatino}

\usepackage{url}
\usepackage{fancyvrb}

\newcommand{\captionfont}{\footnotesize}

\selectcolormodel{cmyk}

\graphicspath{{images/}}

%%%%%%%%%%%%%%%%%%%%%%%%%%%%%%%%%%%%%%%%%%%%%%%%%%%%%%%%%%%%%%%%%%%%%%%%%%%%%%%%
%%%% Some math symbols used in the text
%%%%%%%%%%%%%%%%%%%%%%%%%%%%%%%%%%%%%%%%%%%%%%%%%%%%%%%%%%%%%%%%%%%%%%%%%%%%%%%%
% Format 
\newcommand{\Matrix}[1]{\begin{bmatrix} #1 \end{bmatrix}}
\newcommand{\Vector}[1]{\Matrix{#1}}
\newcommand*{\SET}[1]  {\ensuremath{\mathcal{#1}}}
\newcommand*{\MAT}[1]  {\ensuremath{\mathbf{#1}}}
\newcommand*{\VEC}[1]  {\ensuremath{\bm{#1}}}
\newcommand*{\CONST}[1]{\ensuremath{\mathit{#1}}}
\newcommand*{\norm}[1]{\mathopen\| #1 \mathclose\|}% use instead of $\|x\|$
\newcommand*{\abs}[1]{\mathopen| #1 \mathclose|}% use instead of $\|x\|$
\newcommand*{\absLR}[1]{\left| #1 \right|}% use instead of $\|x\|$

\def\norm#1{\mathopen\| #1 \mathclose\|}% use instead of $\|x\|$
\newcommand{\normLR}[1]{\left\| #1 \right\|}% use instead of $\|x\|$

%%%%%%%%%%%%%%%%%%%%%%%%%%%%%%%%%%%%%%%%%%%%%%%%%%%%%%%%%%%%%%%%%%%%%%%%%%%%%%%%
% Multicol Settings
%%%%%%%%%%%%%%%%%%%%%%%%%%%%%%%%%%%%%%%%%%%%%%%%%%%%%%%%%%%%%%%%%%%%%%%%%%%%%%%%
\setlength{\columnsep}{0.7em}
\setlength{\columnseprule}{0mm}


%%%%%%%%%%%%%%%%%%%%%%%%%%%%%%%%%%%%%%%%%%%%%%%%%%%%%%%%%%%%%%%%%%%%%%%%%%%%%%%%
% Save space in lists. Use this after the opening of the list
%%%%%%%%%%%%%%%%%%%%%%%%%%%%%%%%%%%%%%%%%%%%%%%%%%%%%%%%%%%%%%%%%%%%%%%%%%%%%%%%
\newcommand{\compresslist}{%
\setlength{\itemsep}{1pt}%
\setlength{\parskip}{0pt}%
\setlength{\parsep}{0pt}%
}

%\newenvironment{code}{\minipage{200em}\verbatim}{\endverbatim\endminipage}
\newenvironment{code}{\verbatim}{\endverbatim}



%%%%%%%%%%%%%%%%%%%%%%%%%%%%%%%%%%%%%%%%%%%%%%%%%%%%%%%%%%%%%%%%%%%%%%%%%%%%%%
%%% Begin of Document
%%%%%%%%%%%%%%%%%%%%%%%%%%%%%%%%%%%%%%%%%%%%%%%%%%%%%%%%%%%%%%%%%%%%%%%%%%%%%%

\begin{document}

%%%%%%%%%%%%%%%%%%%%%%%%%%%%%%%%%%%%%%%%%%%%%%%%%%%%%%%%%%%%%%%%%%%%%%%%%%%%%%
%%% Here starts the poster
%%%---------------------------------------------------------------------------
%%% Format it to your taste with the options
%%%%%%%%%%%%%%%%%%%%%%%%%%%%%%%%%%%%%%%%%%%%%%%%%%%%%%%%%%%%%%%%%%%%%%%%%%%%%%
% Define some colors
\definecolor{silver}{cmyk}{0,0,0,0.3}
\definecolor{yellow}{cmyk}{0,0,0.9,0.0}
\definecolor{reddishyellow}{cmyk}{0,0.22,1.0,0.0}
\definecolor{black}{cmyk}{0,0,0.0,1.0}
\definecolor{darkYellow}{cmyk}{0,0,1.0,0.5}
\definecolor{darkSilver}{cmyk}{0,0,0,0.1}

\definecolor{lightyellow}{cmyk}{0,0,0.3,0.0}
\definecolor{lighteryellow}{cmyk}{0,0,0.1,0.0}
\definecolor{lighteryellow}{cmyk}{0,0,0.1,0.0}
\definecolor{lightestyellow}{cmyk}{0,0,0.05,0.0}

\definecolor{white}{cmyk}{0,0,0,0}
\definecolor{gray5}{cmyk}{0,0,0,0.05}
\definecolor{gray30}{cmyk}{0,0,0,0.3}
\definecolor{gray50}{cmyk}{0,0,0,0.5}
\definecolor{gray90}{cmyk}{0,0,0,0.9}

%%
\typeout{Poster Starts}
\background{
  \begin{tikzpicture}[remember picture,overlay]%
    \draw (current page.north west)+(-2em,2em) node[anchor=north west] {\includegraphics[height=1.1\textheight]{silhouettes_background}};
  \end{tikzpicture}%
}

\newlength{\leftimgwidth}
\begin{poster}%
  % Poster Options
  {
  % Show grid to help with alignment
  grid=no,
  % Column spacing
  colspacing=1em,
  % Color style
  bgColorOne=white,
  bgColorTwo=white,
  borderColor=black,
  headerColorOne=gray50,
  headerColorTwo=gray90,
  headerFontColor=reddishyellow,
  boxColorOne=gray5,
  boxColorTwo=gray5,
  % Format of textbox
  textborder=none, %rectangle,
  textfont=\sf, %Sans Serif
 % Format of text header
  eyecatcher=no,
  headerborder=none,
  headerheight=0.08\textheight,
  headershape=rectangle,
  headershade=shade-tb,
  headerfont=\Large\textsf, %Sans Serif
  boxshade=none, %shade-tb,
%  background=shade-tb,
  background=none,
  linewidth=2pt
  }
  % Eye Catcher
  {\includegraphics[width=1em]{nipylogo}} % No eye catcher for this poster. (eyecatcher=no above). If an eye catcher is present, the title is centered between eye-catcher and logo.
  % Title
  {\sf %Sans Serif
  %\bf% Serif
  Nipype: Opensource platform for unified and replicable interaction
  with existing neuroimaging tools\vspace{0.15em}}
  % Authors
  {\sf %Sans Serif
  % Serif
    SS Ghosh$^1$, C Burns$^2$, D Clark$^2$, K Gorgolewski$^3$, YO
    Halchenko$^4$, C Madison$^2$, R Tungaraza$^5$, KJ
    Millman$^2$\\
    \small\sf$^1$MIT, Cambridge, MA, USA $^2$U. California, Berkeley, CA, USA $^3$U. Edinburgh, UK
    $^4$Dartmouth College, Hanover, NH, USA $^5$U. Washington, Seattle, WA, USA\vspace{-0.4em}}

  % University logo
  % {% The makebox allows the title to flow into the logo, this is a hack because of the L shaped logo.
  %   \makebox[8em][r]{%
  %     \begin{minipage}{16em}
  %       \hfill
  %       %\includegraphics[height=2em]{msrlogo}
  %       \includegraphics[height=7.0em]{nipylogo}
  %     \end{minipage}
  %   }
  % }

  \tikzstyle{light shaded}=[top color=baposterBGtwo!30!white,bottom color=baposterBGone!30!white,shading=axis,shading angle=30]

  % Width of left inset image
     \setlength{\leftimgwidth}{0.78em+8.0em}

%%%%%%%%%%%%%%%%%%%%%%%%%%%%%%%%%%%%%%%%%%%%%%%%%%%%%%%%%%%%%%%%%%%%%%%%%%%%%%
%%% Now define the boxes that make up the poster
%%%---------------------------------------------------------------------------
%%% Each box has a name and can be placed absolutely or relatively.
%%% The only inconvenience is that you can only specify a relative position 
%%% towards an already declared box. So if you have a box attached to the 
%%% bottom, one to the top and a third one which should be in between, you 
%%% have to specify the top and bottom boxes before you specify the middle 
%%% box.
%%%%%%%%%%%%%%%%%%%%%%%%%%%%%%%%%%%%%%%%%%%%%%%%%%%%%%%%%%%%%%%%%%%%%%%%%%%%%%
    %
    % A coloured circle useful as a bullet with an adjustably strong filling
    \newcommand{\colouredcircle}[1]{%
      \tikz{\useasboundingbox (-0.2em,-0.32em) rectangle(0.2em,0.32em); \draw[draw=black,fill=headerFontColor!80!black!#1!white,line width=0.03em] (0,0) circle(0.18em);}}

%%%%%%%%%%%%%%%%%%%%%%%%%%%%%%%%%%%%%%%%%%%%%%%%%%%%%%%%%%%%%%%%%%%%%%%%%%%%%%
  \headerbox{Introduction}{name=introduction,column=0,row=0}{
%%%%%%%%%%%%%%%%%%%%%%%%%%%%%%%%%%%%%%%%%%%%%%%%%%%%%%%%%%%%%%%%%%%%%%%%%%%%%%
    Current neuroimaging software offer users an incredible opportunity
    to analyze their data in different ways, with different underlying
    assumptions. However, this has resulted in a heterogeneous
    collection of specialized applications without transparent
    interoperability or a uniform operating interface. Nipype solves
    these issues by providing a uniform interface to existing
    neuroimaging software and by facilitating interaction between these
    packages within a single workflow.
 }

%%%%%%%%%%%%%%%%%%%%%%%%%%%%%%%%%%%%%%%%%%%%%%%%%%%%%%%%%%%%%%%%%%%%%%%%%%%%%%
  \headerbox{Problems}{name=problems,column=0,below=introduction}{
%%%%%%%%%%%%%%%%%%%%%%%%%%%%%%%%%%%%%%%%%%%%%%%%%%%%%%%%%%%%%%%%%%%%%%%%%%%%%%
\colouredcircle{100} \hspace{0.1em}\textbf{Optimal workflows.}
Integrating different packages is tedious. 

\colouredcircle{100} \hspace{0.1em}\textbf{Developers nightmare.} No
standard framework for disseminating tools across packages.

\colouredcircle{100} \hspace{0.1em}\textbf{Knowledge distribution.} Training new
personnel takes time.

\colouredcircle{100} \hspace{0.1em}\textbf{Leveraging technology.} Many
packages do not address computational efficiency.

\colouredcircle{100} \hspace{0.1em}\textbf{Replicating research.} Method
sections are often inadequate for replication.
}

%%%%%%%%%%%%%%%%%%%%%%%%%%%%%%%%%%%%%%%%%%%%%%%%%%%%%%%%%%%%%%%%%%%%%%%%%%%%%%
  \headerbox{Features}{name=features,column=0,below=problems}{
%%%%%%%%%%%%%%%%%%%%%%%%%%%%%%%%%%%%%%%%%%%%%%%%%%%%%%%%%%%%%%%%%%%%%%%%%%%%%%
\colouredcircle{100} \hspace{0.1em}\textbf{Multi-package support.} Uniform
cross-package interface (supports SPM, FSL, FreeSurfer components). 

\colouredcircle{100} \hspace{0.1em}\textbf{Minimal programming knowledge.} A high
level scripting interface to design workflows intended for non-programmers.

\colouredcircle{100} \hspace{0.1em}\textbf{Parallel execution.} No additional coding
necessary for parallel execution of processing stages.

\colouredcircle{100} \hspace{0.1em}\textbf{Pedagogical.} Can be used as a teaching
tool and facilitates user training.
 
\colouredcircle{100} \hspace{0.1em}\textbf{Well documented.} Code documentation
becomes user documentation.

\colouredcircle{100} \hspace{0.1em}\textbf{Shared workflows.} Workflows separate
execution stages from data and can be shared within and across labs.
}

%%%%%%%%%%%%%%%%%%%%%%%%%%%%%%%%%%%%%%%%%%%%%%%%%%%%%%%%%%%%%%%%%%%%%%%%%%%%%%
  \headerbox{Component Architecture Diagram}{name=cad,column=1,span=2}{
%%%%%%%%%%%%%%%%%%%%%%%%%%%%%%%%%%%%%%%%%%%%%%%%%%%%%%%%%%%%%%%%%%%%%%%%%%%%%%
  \begin{tikzpicture}[x=\linewidth,y=\linewidth]
    %\path [use as bounding box] (-0.5,-0.5) rectangle(2.5,1.7);
    \node{\includegraphics[width=\linewidth]{nipype_architecture}};
 \end{tikzpicture}
 }

\begin{SaveVerbatim}{InterfaceVerb}
In [5]: from nipype.interfaces.fsl import BET
In [6]: BET.help()

Inputs
------
Mandatory:
 in_file: input file to skull strip

Optional:
 args: Additional parameters to the command
 environ: Environment variables (default={})
 frac: fractional intensity threshold
 functional: apply to 4D fMRI data
  mutually exclusive: functional, reduce_bias
 mask: create binary mask image
...

Outputs
-------
out_file: path/name of skullstripped file
...

In [7]: bet = BET()
In [8]: res = bet.run(in_file='struct.nii', mask=True)
\end{SaveVerbatim}

%%%%%%%%%%%%%%%%%%%%%%%%%%%%%%%%%%%%%%%%%%%%%%%%%%%%%%%%%%%%%%%%%%%%%%%%%%%%%%
  \headerbox{Uniform package interface}{name=interface,column=1,below=cad}{
%%%%%%%%%%%%%%%%%%%%%%%%%%%%%%%%%%%%%%%%%%%%%%%%%%%%%%%%%%%%%%%%%%%%%%%%%%%%%%
    \UseVerbatim[fontsize=\relsize{-2.5}]{InterfaceVerb}
}

\begin{SaveVerbatim}{WorkflowVerbImport}
from nipype.pipeline.engine import (Workflow,
                                    Node, Mapnode)
from nipype.interfaces.spm import Realign
from nipype.interfaces.fsl import Smooth
\end{SaveVerbatim}

\begin{SaveVerbatim}{WorkflowVerbDefine}
realign = Node(interface=Realign(), name='realign')
smooth = MapNode(interface=Smooth(fwhm=6),
                 iterfield=['in_file'], name='smooth')
\end{SaveVerbatim}

\begin{SaveVerbatim}{WorkflowVerbConnect}
preproc = Workflow()
preproc.connect(realign, 'realigned_files',
                smooth, 'in_file')
\end{SaveVerbatim}

\begin{SaveVerbatim}{WorkflowVerbExecute}
preproc.inputs.realign.in_files = ['run1.nii', 'run2.nii']
preproc.base_dir = './workdir'
preproc.run()
\end{SaveVerbatim}

%%%%%%%%%%%%%%%%%%%%%%%%%%%%%%%%%%%%%%%%%%%%%%%%%%%%%%%%%%%%%%%%%%%%%%%%%%%%%%
  \headerbox{Simple workflow construction}{name=workflow,column=2,below=cad}{
%%%%%%%%%%%%%%%%%%%%%%%%%%%%%%%%%%%%%%%%%%%%%%%%%%%%%%%%%%%%%%%%%%%%%%%%%%%%%%
    \textbf{Import components}
    \vspace{-0.3em}
    \UseVerbatim[fontsize=\relsize{-2.5}]{WorkflowVerbImport}
    \vspace{-0.3em}
    \textbf{Define prcoesses}
    \vspace{-0.3em}
    \UseVerbatim[fontsize=\relsize{-2.5}]{WorkflowVerbDefine}
    \vspace{-0.3em}
    \textbf{Construct workflow}
    \vspace{-0.3em}
    \UseVerbatim[fontsize=\relsize{-2.5}]{WorkflowVerbConnect}
    \vspace{-0.3em}
    \textbf{Execute workflow}
    \vspace{-0.3em}
    \UseVerbatim[fontsize=\relsize{-2.5}]{WorkflowVerbExecute}
}

%%%%%%%%%%%%%%%%%%%%%%%%%%%%%%%%%%%%%%%%%%%%%%%%%%%%%%%%%%%%%%%%%%%%%%%%%%%%%%
  \headerbox{Funding}{name=funding,column=1,span=2,above=bottom}{
%%%%%%%%%%%%%%%%%%%%%%%%%%%%%%%%%%%%%%%%%%%%%%%%%%%%%%%%%%%%%%%%%%%%%%%%%%%%%%
    \smaller This project was supported by NIH grants NIBIB R03 EB008673
    (PI: Ghosh, Whitfield-Gabrieli), NIMH R01 MH081909 (PI: D'Esposito).
  }

%%%%%%%%%%%%%%%%%%%%%%%%%%%%%%%%%%%%%%%%%%%%%%%%%%%%%%%%%%%%%%%%%%%%%%%%%%%%%%
  \headerbox{Hierarchical workflow visualization}{name=hierarchy,column=1,span=2,below=interface}{
%%%%%%%%%%%%%%%%%%%%%%%%%%%%%%%%%%%%%%%%%%%%%%%%%%%%%%%%%%%%%%%%%%%%%%%%%%%%%%
  \begin{tikzpicture}[x=\linewidth,y=\linewidth]
    %\draw[help lines] (0,0) grid (0.5,0.5);
    \node at (-0.2,0) {\includegraphics[width=0.2\linewidth]{level1}};
    \node at (0,0) {\includegraphics[width=0.3\linewidth]{firstlevel}};
    \node at (0.42,-0.01) {\includegraphics[width=0.6\linewidth]{detailed_graph}};
    \draw (0.13,-0.2) rectangle(0.71,0.2);
    \draw (0.18, 0.185) node(preproc) {preproc};
    \draw (-0.125,-0.2) rectangle(0.125,0.2);
    \draw (-0.075, 0.185) node(firstlevel) {firstlevel};
 \end{tikzpicture}
 \vspace{-1em}
 }

%%%%%%%%%%%%%%%%%%%%%%%%%%%%%%%%%%%%%%%%%%%%%%%%%%%%%%%%%%%%%%%%%%%%%%%%%%%%%%
  \headerbox{Parallel execution}{name=parallel,column=1,below=hierarchy}{
%%%%%%%%%%%%%%%%%%%%%%%%%%%%%%%%%%%%%%%%%%%%%%%%%%%%%%%%%%%%%%%%%%%%%%%%%%%%%%
   \colouredcircle{100} \hspace{0.1em}Leverages IPython for distributed computation. 

   \colouredcircle{100} \hspace{0.1em}Non-dependent processes can be executed in parallel.

   \colouredcircle{100} \hspace{0.1em}No additional coding necessary.

   \colouredcircle{100} \hspace{0.1em}Example. 

   \hspace{0.5cm}\colouredcircle{100} \hspace{0.1em}SPM fMRI workflow on
   79 subjects

   \hspace{0.5cm}\colouredcircle{100} \hspace{0.1em}Typical runtime: 3 days

   \hspace{0.5cm}\colouredcircle{100} \hspace{0.1em}Nipype runtime: 1.67
   hrs on 40 cores.  }

%%%%%%%%%%%%%%%%%%%%%%%%%%%%%%%%%%%%%%%%%%%%%%%%%%%%%%%%%%%%%%%%%%%%%%%%%%%%%%
  \headerbox{Future plans}{name=future,column=2,below=hierarchy}{
%%%%%%%%%%%%%%%%%%%%%%%%%%%%%%%%%%%%%%%%%%%%%%%%%%%%%%%%%%%%%%%%%%%%%%%%%%%%%%
    
    \colouredcircle{100} \hspace{0.1em}\textbf{More interfaces.} For
    example, newer registration algorithms (e.g., ANTS - U. Penn, CVS -
    MGH), integration with statistical tools in NiPy.

    \colouredcircle{100} \hspace{0.1em}\textbf{Web interface.} A
    graphical interface to define workflows.

    \colouredcircle{100} \hspace{0.1em}\textbf{Workflow repository.}
    Create a centralized location for version controlled workflows. 

   \colouredcircle{100} \hspace{0.1em}\textbf{Additional modalities.}
    Extend to PET, EEG/MEG, fNIRS.

}

%%%%%%%%%%%%%%%%%%%%%%%%%%%%%%%%%%%%%%%%%%%%%%%%%%%%%%%%%%%%%%%%%%%%%%%%%%%%%%
  \headerbox{Sotware}{name=software,column=0,above=bottom}{
%%%%%%%%%%%%%%%%%%%%%%%%%%%%%%%%%%%%%%%%%%%%%%%%%%%%%%%%%%%%%%%%%%%%%%%%%%%%%%
   \begin{tabular}{ll}
     NIPY & \url{nipy.sourceforge.net}\\
     Ipython & \url{ipython.scipy.org}\\
     Scipy,Numpy & \url{scipy.org}\\
     Networkx & \url{networkx.lanl.gov}\\
     NeuroDebian & \url{neurodebian.net}\\
     Traits & \url{www.enthought.com}
   \end{tabular}

 }

%%%%%%%%%%%%%%%%%%%%%%%%%%%%%%%%%%%%%%%%%%%%%%%%%%%%%%%%%%%%%%%%%%%%%%%%%%%%%%
  \headerbox{Conclusion}{name=conclusion,column=0,span=1,below=features,above=software}{
%%%%%%%%%%%%%%%%%%%%%%%%%%%%%%%%%%%%%%%%%%%%%%%%%%%%%%%%%%%%%%%%%%%%%%%%%%%%%%
    \colouredcircle{100} \hspace{0.1em}Provides an environment for
    interactive manipulation of data through a Python interface as well
    as for performing reproducible, distributed analysis using a
    pipeline system.

    \colouredcircle{100} \hspace{0.1em}Encourages scientific exploration
    of different algorithms and associated parameters.

    \colouredcircle{100} \hspace{0.1em}Simplifies the development of
    workflows within and between packages.

    \colouredcircle{100} \hspace{0.1em}Reduces the learning curve
    associated with understanding the algorithms, APIs and user
    interfaces of disparate packages.

   \begin{tabular}{ll}
      & \\
      \textbf{C}ollaborative & Engages the community\\
      \textbf{O}pensource & Available for all\\
      \textbf{R}eproducible & Workflows are records\\
      \textbf{E}fficient & Minimize redundancy
    \end{tabular}
}

\end{poster}%
\end{document}
